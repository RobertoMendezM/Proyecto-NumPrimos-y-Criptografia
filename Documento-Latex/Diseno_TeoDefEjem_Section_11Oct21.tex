%
%  Para los marcos de teoremas, ejemplos,
%					y definiciones.
% 
%  \begin{D1ejemplo}[-algún texto].
%
%  \end{D1ejemplo}

%  Editor: Roberto Méndez Méndez
%  Última edición: Agosto 2015
%        --   2-Noviembere-2018   (hay otra mejor) --
%           15-Abr-2020
%          11-Oct-2021  
%  Nombre anterior: Diseno_TeoDefEjem.tex


\usepackage[shortlabels]{enumitem}




\usepackage{tikz}
%\usetikzlibrary{tikzmark} NUNCA SE UTILIZA
%\usepackage{tkz-tab}% EN DESHUSO
\usetikzlibrary{matrix,arrows, positioning,shadows,shadings,backgrounds,
calc}

\usepackage{tcolorbox, empheq, xparse}%
\tcbuselibrary{skins,breakable,listings,theorems}


%   FORMATO TEOREMAS 

\definecolor{colordominanteD}{RGB}{74,0,148}

\newcounter{tcbteo}[section] % Contador
 %\renewcommand{\thetcbteo}{\thesection.\arabic{tcbteo}}% Formato 1.1,1.2
\renewcommand{\thetcbteo}{\arabic{tcbteo})}


% Estilo "nodoTeorema" para nodos
\tikzset{nodoTeorema/.style={%
rectangle, top color=gray!5, bottom color=gray!5,
inner sep=1mm,anchor=west,font=\small\bf\sffamily}
}
% Caja de entorno
\newtcolorbox{cajaTeorema}[3][]{%
% Opciones generales
arc=0mm,breakable,enhanced,colback=gray!5,boxrule=0pt,top=7mm,
drop fuzzy shadow, fontupper=\normalsize,
% label
step and label={tcbteo}{#3},
interior style={color=yellow!10!white},
overlay unbroken= {%
% Borde superior grueso.
% "--+(0pt,3pt)" significa: 3pt hacia arriba desde la posición anterior
\draw[colordominanteD,line width =2.5cm]
([xshift=1.25cm, yshift=0cm]frame.north west)--+(0pt,3pt);
% Borde superior 1
\draw[color=colordominanteD,line width =0.2pt]
(frame.north west)--([xshift=0pt]frame.north east);
% Caja Teorema-contador
\node[nodoTeorema](tituloteo)
at ([xshift=0.2cm, yshift=-4mm]frame.north west)
{\textbf{\color{colordominanteD} Teorema \thetcbteo \;#2}};
},
overlay first = {
% Borde superior grueso
\draw[colordominanteD,line width =2.5cm]
([xshift=1.25cm, yshift=0cm]frame.north west)--+(0pt,3pt);
% Borde superior 1
\draw[color=colordominanteD,line width =0.2pt]
(frame.north west)--([xshift=0pt]frame.north east);
% Caja Teorema-contador
\node[nodoTeorema](tituloteo)
at ([xshift=0.2cm, yshift=-4mm]frame.north west)
{\textbf{\color{colordominanteD} Teorema \thetcbteo \;#2}};
}, %First
% Nada que mantener en los cambios de página
overlay middle = { },
overlay last = { }
#1}

%- Uso \begin{teorema}... o \begin{teorema}[de tal] o \begin{teorema}[][ref]
\NewDocumentEnvironment{teoremaf2}{O{} O{} O{}}{
\bigskip\begin{cajaTeorema}{#1}{#2}%
#3
}{\end{cajaTeorema}\bigskip } 
%%%%%%%%%%%%%%%%% FIN FORMATO TEOREMA


%%%%%%%%%%%    FORMATO EJEMPLO

\newcounter{tcbejem}[section]
% \renewcommand{\thetcbejem}{\thesection.\arabic{tcbejem}} 
\renewcommand{\thetcbejem}{\arabic{tcbejem})}

\definecolor{colordominanteEj}{RGB}{34,159,159}

\tikzset{
    wnodeTeorema/.style={%
         rectangle,  top color=gray!5, bottom color=gray!5,
         inner sep=1mm,anchor=west,font=\small\bf\sffamily},
   wnodeminimo/.style={%
         rectangle,  top color=white, bottom color=white,
         text=azulF,inner sep=1mm,anchor=west,font=\small\bf\sffamily}      
}

% % Ejemplo---------------------------------------------------
\newtcolorbox{wwejemplo}[3][]{%
arc=0mm,breakable,enhanced,colback=gray!5,boxrule=0pt,
top=6mm,fontupper=\normalsize,step and label={tcbejem}{#3},
interior style={color=blue!3!white},
overlay unbroken  = {
%borde
\draw[color=gray!20,line width=0.2pt] (frame.north west)
  --([xshift=0pt]frame.north east)
  --([xshift=0pt]frame.south east);
%  --([xshift=0pt]frame.south west)--(frame.north west);
%barra vertical
\draw[color=colordominanteEj,line width=3pt] ([xshift=2pt] frame.north west)--([xshift=2pt] frame.south west);          
%Caja de Título: defi --
\node[wnodeTeorema](titulodefi) at ([xshift=4.5pt, yshift=-3mm]frame.north west)
{\textbf{Ejemplo \thetcbejem \;#2}};
                }, %overlay
overlay first  = {
%borde
\draw[color=gray!20,line width=0.2pt] (frame.north west)
  --([xshift=0pt]frame.north east)
  --([xshift=0pt]frame.south east);
%  --([xshift=0pt]frame.south west)--(frame.north west);
%barra vertical
\draw[color=colordominanteEj,line width=3pt] ([xshift=2pt] frame.north west)--([xshift=2pt] frame.south west);          
%Caja de Título: defi --
\node[wnodeTeorema](titulodefi) at ([xshift=4.5pt, yshift=-3mm]frame.north west)
{\textbf{Ejemplo \thetcbejem \;#2}};
                }, %overlay
% Mantener borde en cambio de página
overlay last    = {
%borde
\draw[color=gray!20,line width=0.2pt] (frame.north west)
  --([xshift=0pt]frame.north east)
  --([xshift=0pt]frame.south east);
%  --([xshift=0pt]frame.south west)--(frame.north west);
%barra vertical
\draw[color=colordominanteEj,line width=3pt] ([xshift=2pt] frame.north west)--([xshift=2pt] frame.south west);}
#1}
%-
\NewDocumentEnvironment{ejemplof2}{O{} O{} O{}}{\smallskip\begin{wwejemplo}{#1}{#2}%
 #3}{\end{wwejemplo}\smallskip }
 
% %FIN EJEMPLO ----------------------

%%%%%%%%%%%    FORMATO DEFINICION
\newcounter{tcbdefi}[section]
%\renewcommand{\thetcbdefi}{\thesection.\arabic{tcbdefi}}
\renewcommand{\thetcbdefi}{\arabic{tcbdefi})}

\definecolor{colordominante}{RGB}{243,102,25}

\tikzset{
    wnodeTeorema/.style={%
         rectangle,  top color=gray!5, bottom color=gray!5,
         inner sep=1mm,anchor=west,font=\small\bf\sffamily},
   wnodeminimo/.style={%
         rectangle,  top color=white, bottom color=white,
         text=azulF,inner sep=1mm,anchor=west,font=\small\bf\sffamily}      
}

% % Definición---------------------------------------------------
\newtcolorbox{wwdefinicion}[3][]{%
arc=0mm,breakable,enhanced,colback=gray!5,boxrule=0pt,
top=6mm,fontupper=\normalsize,step and label={tcbdefi}{#3},
interior style={color=blue!5!white},
overlay unbroken  = {
%borde
\draw[color=gray!20,line width=0.2pt] (frame.north west)
  --([xshift=0pt]frame.north east)
  --([xshift=0pt]frame.south east);
%  --([xshift=0pt]frame.south west)--(frame.north west);
%barra vertical
\draw[color=colordominante,line width=3pt] ([xshift=2pt] frame.north west)--([xshift=2pt] frame.south west);          
%Caja de Título: defi --
\node[wnodeTeorema](titulodefi) at ([xshift=4.5pt, yshift=-3mm]frame.north west)
{\textbf{Definición \thetcbdefi \;#2}};
                }, %overlay
overlay first  = {
%borde
\draw[color=gray!20,line width=0.2pt] (frame.north west)
  --([xshift=0pt]frame.north east)
  --([xshift=0pt]frame.south east);
%  --([xshift=0pt]frame.south west)--(frame.north west);
%barra vertical
\draw[color=colordominante,line width=3pt] ([xshift=2pt] frame.north west)--([xshift=2pt] frame.south west);          
%Caja de Título: defi --
\node[wnodeTeorema](titulodefi) at ([xshift=4.5pt, yshift=-3mm]frame.north west)
{\textbf{Definición \thetcbdefi \;#2}};
                }, %overlay
% Mantener borde en cambio de página
overlay last    = {
%borde
\draw[color=gray!20,line width=0.2pt] (frame.north west)
  --([xshift=0pt]frame.north east)
  --([xshift=0pt]frame.south east);
%  --([xshift=0pt]frame.south west)--(frame.north west);
%barra vertical
\draw[color=colordominante,line width=3pt] ([xshift=2pt] frame.north west)--([xshift=2pt] frame.south west);}
#1}
%-
\NewDocumentEnvironment{definicionf2}{O{} O{} O{}}{\smallskip\begin{wwdefinicion}{#1}{#2}%
 #3}{\end{wwdefinicion}\smallskip }
% %DEFINICION----------------------



